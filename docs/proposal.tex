\documentclass{article}

% Language setting
% Replace `english' with e.g. `spanish' to change the document language
\usepackage[english]{babel}

% Set page size and margins
% Replace `letterpaper' with `a4paper' for UK/EU standard size
\usepackage[letterpaper,top=2cm,bottom=2cm,left=3cm,right=3cm,marginparwidth=1.75cm]{geometry}

% Useful packages
\usepackage{amsmath}
\usepackage{graphicx}
\usepackage[colorlinks=true, allcolors=blue]{hyperref}

\title{SafeCode: IDE plugin for JavaScript developers to provide secure coding assistance}
\author{You}

\begin{document}
\maketitle

\begin{abstract}
This project aims to develop an Integrated Development Environment (IDE) plugin to improve developers' security awareness and practices when writing JavaScript code. The plugin will point out security vulnerabilities in code in real time and provide solutions and best practices through a user-friendly interface. The target users are non-computer students, developers who do not have security as a priority, and junior developers who want to learn secure development.
\end{abstract}

\section{Introduction}

\subsection{Background}
With the popularity of the Internet, front-end development and JavaScript are becoming more and more popular. However, many developers tend to focus on performance at the expense of security when writing code, especially non-computer professionals and developers who lack security awareness. In addition, existing security tools are often not user-friendly enough and difficult for non-experts to understand.

\subsection{Purpose}
To develop an IDE plug-in that can point out possible security threats in real time during development, providing easy-to-understand information and solutions through an intuitive interface, thereby improving the security awareness and capabilities of JavaScript developers.

\section{Objectives and Scope}

\subsection{Main Objectives}
\begin{enumerate}
    \item Detect common security vulnerabilities (e.g., SQL injection, XSS, CSRF, etc.) in JavaScript code in real time.
    \item Provide clear and intuitive visual feedback on detected security issues.
    \item Provide solutions to problems and best practices.
    \item Optimize user experience for non-computer professionals and junior developers.
\end{enumerate}

\begin{itemize}
    \item Focus on JavaScript programming language, including the Vue.js framework and backend Node.js.
    \item Target common security issues and vulnerabilities.
    \item Target the Visual Studio Code IDE.
\end{itemize}

\section{Methods and Methodology}

\begin{enumerate}
    \item \textbf{Market Research and Analysis}: Conduct market research on existing security tools, analyze their strengths and weaknesses, and develop the core functions and features of our plugin based on this.
    \item \textbf{Technology Selection}: Select suitable technologies and tools to develop the IDE plugin.
    \item \textbf{Security Rule Development}: Develop a set of custom detection rules for identifying common security issues in JavaScript code.
    \item \textbf{User Interface Design}: Design user-friendly interfaces that provide intuitive visual feedback and suggestions.
    \item \textbf{Plugin Development}: Code the plugin based on preliminary research and design.
    \item \textbf{Testing and Evaluation}: Test the functionality and performance of the plugin on different IDEs and code samples, and optimize based on feedback.
\end{enumerate}

\section{Expected Results}

\begin{itemize}
    \item An IDE plugin capable of detecting JavaScript code security issues in real time.
    \item An intuitive user interface that provides easy-to-understand feedback and solutions.
    \item A detailed project report that includes market research, development process, and evaluation results.
\end{itemize}

\section{Timeline}

\begin{itemize}
    \item Week 1: Market research and analysis
    \item Week 2: Technology selection
    \item Week 3: Development of security rules
    \item Week 4: User interface design
    \item Weeks 5-6: Plugin development
    \item Week 7: Testing and evaluation
    \item Week 8: Writing project report
    \item Weeks 9-10: Submit final project report and plugin
\end{itemize}

\bibliographystyle{alpha}
\bibliography{sample}

\end{document}
